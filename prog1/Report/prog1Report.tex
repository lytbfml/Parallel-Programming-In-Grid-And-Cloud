\documentclass[11pt, letterpaper]{article}
\usepackage{color}
\usepackage[linktoc=all]{hyperref}

\usepackage[bindingoffset=0.2in,left=1in,right=1in,top=0.5in,bottom=1in,footskip=.25in]{geometry}
\usepackage{enumitem}
\usepackage{graphicx}
\usepackage{subcaption}
\usepackage{mwe}
\usepackage{parskip}
\usepackage{indentfirst}
\usepackage{textcomp}
\usepackage[formats]{listings}
\usepackage{xcolor}
\definecolor{listinggray}{gray}{0.9}
\definecolor{lbcolor}{rgb}{0.9,0.9,0.9}
\definecolor{Darkgreen}{RGB}{11,100,35}

\hypersetup{
	colorlinks,
	citecolor=black,
	filecolor=black,
	linkcolor=black,
	urlcolor=black
}

\lstset{
	backgroundcolor=\color{lbcolor},
	tabsize=4,    
	%   rulecolor=,
	language=[GNU]C++,
	basicstyle=\scriptsize,
	upquote=true,
	aboveskip={1.5\baselineskip},
	columns=fixed,
	showstringspaces=false,
	extendedchars=false,
	breaklines=true,
	prebreak = \raisebox{0ex}[0ex][0ex]{\ensuremath{\hookleftarrow}},
	frame=single,
	numbers=left,
	showtabs=false,
	showspaces=false,
	showstringspaces=false,
	identifierstyle=\ttfamily,
	keywordstyle=\color[rgb]{0,0,1},
	commentstyle=\color[rgb]{0.026,0.112,0.095},
	stringstyle=\color[rgb]{0.627,0.126,0.941},
	numberstyle=\color[rgb]{0.205, 0.142, 0.73},
	%        \lstdefinestyle{C++}{language=C++,style=numbers}’.
}



\lstset{
	backgroundcolor=\color{lbcolor},
	tabsize=4,
	language=C++,
	captionpos=b,
	tabsize=3,
	frame=lines,
	numbers=left,
	numberstyle=\tiny,
	numbersep=5pt,
	breaklines=true,
	showstringspaces=false,
	basicstyle=\footnotesize,
	%  identifierstyle=\color{magenta},
	keywordstyle=\color[rgb]{0,0,1},
	commentstyle=\color{Darkgreen},
	stringstyle=\color{red}
}



\setlength{\parindent}{15pt}
%opening
\title{Program 1 Report}
\author{Yangxiao Wang}
\date{ }



\begin{document}
	
	\maketitle
	
	\tableofcontents
	\pagebreak
	
	\section{Documentation}
	In this project, we tried to solve Traveling salesman problem (TSP) with genetic algorithm. We are required to implement the core part of the algorithm: evaluation, crossover and mutation. Tsp problem often takes large amount of data and time. And to improve the efficiency, our approach is using OpenMP. My initial attempt is implement a greedy parallelization: parallelize the whole program from \textit{evaluate()} to \textit{populate()} with: \begin{lstlisting}
	#pragma omp parallel for
	\end{lstlisting}
	however, after trying different things with the program, I realized that populate and mutate does not need to be parallelized. The time spent did increase not have much difference. And \textit{select()} with parallelization could spend more time than when running multi-thread. Therefore, the only functions I need to parallelize are \textit{crossover()} and \textit{evaluate()} which contain much larger loops. I simply added \begin{small}
	\#pragma omp parallel for
	\end{small} before the outer for loop inside \textit{crossover()} and \textit{evaluate()}. \par
	
	
	
	\section {Source code}
	\subsection{EvalXOverMutate.cpp}
	\begin{lstlisting}
	//
	// Created by yangxiao on 10/8/2018.
	//
	#include <iostream>  // cout
	#include <fstream>   // ifstream
	#include <string.h>  // strncpy
	#include <stdlib.h>  // rand
	#include <math.h>    // sqrt, pow
	#include <time.h>
	#include <omp.h>
	#include <bits/stdc++.h>
	#include "Trip.h"
	
	using namespace std;
	
	
	float getDis(char a, char b, int coordinates[CITIES][2]);
	
	float getDisFromIt(char str[CITIES], float disMax[CITIES][CITIES + 1]);
	
	char *getChildB(char childA[CITIES]);
	
	int getIndex(char a, char source[CITIES]);
	
	int getAscii(char source);
	
	const char cit[] = "ABCDEFGHIJKLMNOPQRSTUVWXYZ0123456789";
	
	/**
	* Compare tow trip based their fitness, uses as comparator for sorting
	* @param t1 trip 1
	* @param t2 trip 2
	* @return if t1 < t2
	*/
	bool compareTrip(Trip t1, Trip t2) {
	return (t1.fitness < t2.fitness);
	}
	
	/**
	* evaluates the distance of each trip and sorts out all the trips 
	* in the shortest-first order
	* @param trip trips to evaluates
	* @param coordinates coordinates of all cities
	*/
	void evaluate(Trip trip[CHROMOSOMES], int coordinates[CITIES][2]) {
	
	//Parallelization
	#pragma omp parallel for
	for (int i = 0; i < CHROMOSOMES; ++i) {
	char temp[CITIES];
	strncpy(temp, trip[i].itinerary, CITIES);
	int indexIn = (temp[0] >= 'A') ? temp[0] - 'A' : temp[0] - '0' + 26;
	double dis = hypot(coordinates[indexIn][0], coordinates[indexIn][1]);
	for (int j = 1; j < CITIES; ++j) {
	int index = (temp[j] >= 'A') ? temp[j] - 'A' : temp[j] - '0' + 26;
	dis += hypot((coordinates[indexIn][0] - coordinates[index][0]),
	(coordinates[indexIn][1] - coordinates[index][1]));
	indexIn = index;
	}
	trip[i].fitness = (float) dis;
	}
	sort(trip, trip + CHROMOSOMES, compareTrip);
	}
	
	/**
	* generates 25,000 off-springs from the parents, calculate distance
	* based on their coordinates
	* @param parents
	* @param offsprings
	* @param coordinates
	*/
	void crossover(Trip parents[TOP_X], Trip offsprings[TOP_X],
	int coordinates[CITIES][2]) {
	#pragma omp parallel for
	for (int i = 0; i < TOP_X; i += 2) {
	int selected[100] = {0};
	char child1[CITIES + 1];
	char child2[CITIES + 1];
	char *parent1 = parents[i].itinerary;
	char *parent2 = parents[i + 1].itinerary;
	child1[0] = parent1[0];
	selected[child1[0]] = 1;
	
	for (int j = 0; j < (CITIES - 1); j++) {
	int indexInA = getIndex(child1[j], parent1) + 1;
	int indexInB = getIndex(child1[j], parent2) + 1;
	
	if (indexInA == -1 || indexInB == -1) {
	printf("WHat?\n");
	}
	
	if (indexInA >= CITIES) {
	indexInA = 0;
	}
	if (indexInB >= CITIES) {
	indexInB = 0;
	}
	
	float disA = INT_MAX;
	float disB = INT_MAX;
	
	if (selected[parent1[indexInA]] == 0) {
	disA = getDis(child1[j], parent1[indexInA], coordinates);
	}
	if (selected[parent2[indexInB]] == 0) {
	disB = getDis(child1[j], parent2[indexInB], coordinates);
	}
	
	if (disA == INT_MAX && disB == INT_MAX) {
	int r = rand() % CITIES;
	int found = 1;
	while (found) {
	if (selected[parent1[r]] != 1) {
	found = 0;
	child1[j + 1] = parent1[r];
	}
	r = ((r + 1) >= CITIES) ? 0 : (r + 1);
	}
	} else {
	child1[j + 1] = (disA <= disB) ? 
	parent1[indexInA] : parent2[indexInB];
	}
	
	selected[child1[j + 1]] = 1;
	}
	
	
	strcpy(child2, getChildB(child1));
	strncpy(offsprings[i].itinerary, child1, CITIES);
	strncpy(offsprings[i + 1].itinerary, child2, CITIES);
	}
	}
	
	
	/**
	* Improved version, when mutate each city, 
	* perform mutation if the new one has shorter distance,
	* otherwise, do nothing
	*
	* randomly chooses two distinct cities (or genes) 
	* in each trip (or chromosome) with a
	* given probability, and swaps them
	* @param offsprings
	* @param disMax
	*/
	void mutateB(Trip offsprings[TOP_X], float disMax[CITIES][CITIES + 1]) {
	for (int i = 0; i < TOP_X; ++i) {
	int rate = rand() % 100;
	if (rate < MUTATE_RATE) {
	int r1 = rand() % CITIES;
	int r2 = rand() % CITIES;
	char temp[CITIES + 1];
	strncpy(temp, offsprings[i].itinerary, CITIES);
	temp[r1] = offsprings[i].itinerary[r2];
	temp[r2] = offsprings[i].itinerary[r1];
	float d1 = getDisFromIt(offsprings[i].itinerary, disMax);
	float d2 = getDisFromIt(temp, disMax);
	//check if the new one has shorter distacne
	if (d1 > d2) {
	strncpy(offsprings[i].itinerary, temp, CITIES);
	}
	}
	}
	}
	
	/**
	* Helper function that calculate distance between two cities
	*/
	float getDis(char a, char b, int coordinates[CITIES][2]) {
	int indexA = (a >= 'A') ? a - 'A' : a - '0' + 26;
	int indexB = (b >= 'A') ? b - 'A' : b - '0' + 26;
	float dis = (float) hypot((coordinates[indexA][0] - coordinates[indexB][0]),
	(coordinates[indexA][1] - coordinates[indexB][1]));
	return dis;
	}
	
	/**
	* Get distacne of the provided route from distance matrix
	*/
	float getDisFromIt(char str[CITIES], float disMax[CITIES][CITIES + 1]) {
	int indexIn = (str[0] >= 'A') ? str[0] - 'A' : str[0] - '0' + 26;
	double dis = disMax[indexIn][CITIES];
	for (int j = 1; j < CITIES; ++j) {
	int index = (str[j] >= 'A') ? str[j] - 'A' : str[j] - '0' + 26;
	dis += disMax[indexIn][index];
	
	indexIn = index;
	}
	return dis;
	}
	
	/**
	* Get the position of the target character
	* @param a target character
	* @param source source string
	* @return index of target char, -1 if not found
	*/
	int getIndex(char a, char source[CITIES]) {
	for (int i = 0; i < CITIES; ++i) {
	if (a == source[i]) return i;
	}
	return -1;
	}
	
	/**
	* get ascii representation of target char, and map it to its index
	* based on ABCDEFGHIJKLMNOPQRSTUVWXYZ0123456789
	*/
	int getAscii(char source) {
	return (source >= 'A') ? source - 'A' : source - '0' + 26;
	}
	
	/**
	* generate childA's complement
	*/
	char *getChildB(char childA[CITIES]) {
	char *childB = (char *) malloc(sizeof(char) * CITIES);
	for (int i = 0; i < CITIES; ++i) {
	childB[i] = cit[(CITIES - getAscii(childA[i]) - 1)];
	}
	return childB;
	}
	
	
	/**
	* Improved version with distance matrix, avoid calculating distance
	* generates 25,000 off-springs from the parents, 
	* calculate distance based on their coordinates
	* @param parents
	* @param offsprings
	* @param disMax
	*/
	void crossoverB(Trip parents[TOP_X], Trip offsprings[TOP_X], float disMax[CITIES][CITIES + 1]) {
	#pragma omp parallel for default(none) shared(parents, disMax, offsprings)
	for (int i = 0; i < TOP_X; i += 2) {
	int selected[100] = {0};
	char child1[CITIES + 1];
	char child2[CITIES + 1];
	char *parent1 = parents[i].itinerary;
	char *parent2 = parents[i + 1].itinerary;
	child1[0] = parent1[0];
	selected[child1[0]] = 1;
	
	for (int j = 0; j < (CITIES - 1); j++) {
	int indexInA = getIndex(child1[j], parent1) + 1;
	int indexInB = getIndex(child1[j], parent2) + 1;
	
	if (indexInA == -1 || indexInB == -1) {
	printf("WHat?\n");
	}
	
	if (indexInA >= CITIES) {
	indexInA = 0;
	}
	if (indexInB >= CITIES) {
	indexInB = 0;
	}
	
	float disA = INT_MAX;
	float disB = INT_MAX;
	
	if (selected[parent1[indexInA]] == 0) {
	int indA = getAscii(child1[j]);
	int indB = getAscii(parent1[indexInA]);
	disA = disMax[indA][indB];
	}
	if (selected[parent2[indexInB]] == 0) {
	int indA = getAscii(child1[j]);
	int indB = getAscii(parent2[indexInB]);
	disB = disMax[indA][indB];
	}
	
	if (disA == INT_MAX && disB == INT_MAX) {
	int r = rand() % CITIES;
	int found = 0;
	while (!found) {
	if (selected[parent1[r]] != 1) {
	found = 1;
	child1[j + 1] = parent1[r];
	}
	r = ((r + 1) >= CITIES) ? 0 : (r + 1);
	}
	} else {
	child1[j + 1] = (disA <= disB) ? parent1[indexInA] : parent2[indexInB];
	}
	
	selected[child1[j + 1]] = 1;
	}
	
	strcpy(child2, getChildB(child1));
	strncpy(offsprings[i].itinerary, child1, CITIES);
	strncpy(offsprings[i + 1].itinerary, child2, CITIES);
	}
	}
	
	/**
	* randomly chooses two distinct cities (or genes)
	* in each trip (or chromosome) with a
	* given probability, and swaps them
	* @param offsprings
	*/
	void mutate(Trip offsprings[TOP_X]) {
	for (int i = 0; i < TOP_X; ++i) {
	int rate = rand() % 100;
	if (rate < MUTATE_RATE) {
	int r1 = rand() % CITIES;
	int r2 = rand() % CITIES;
	char x1 = offsprings[i].itinerary[r1];
	char x2 = offsprings[i].itinerary[r2];
	offsprings[i].itinerary[r1] = x2;
	offsprings[i].itinerary[r2] = x1;
	}
	}
	}
	
	
	/**
	* Improved verison, uses distance matrix to avoid calculating distance
	* Evaluates the distance of each trip and sorts out all the
	* trips in the shortest-first order
	* @param trip trips to evaluates
	* @param disMax maxtrix that contain all the distance
	* @param first if it's first generation
	*/
	void evaluateB(Trip trip[CHROMOSOMES], 
	float disMax[CITIES][CITIES + 1], bool first) {
	int start = 0;
	if (!first) {
	start = TOP_X;
	}
	
	#pragma omp parallel for default(none) firstprivate(start) shared(trip, disMax)
	for (int i = start; i < CHROMOSOMES; ++i) {
	char temp[CITIES];
	strncpy(temp, trip[i].itinerary, CITIES);
	int indexIn = (temp[0] >= 'A') ? temp[0] - 'A' : temp[0] - '0' + 26;
	double dis = disMax[indexIn][CITIES];
	for (int j = 1; j < CITIES; ++j) {
	int index = (temp[j] >= 'A') ? temp[j] - 'A' : temp[j] - '0' + 26;
	dis += disMax[indexIn][index];
	
	indexIn = index;
	}
	trip[i].fitness = (float) dis;
	//        printf("Thread %d executes loop iteration %d\n", omp_get_thread_num(), i);
	}
	sort(trip, trip + CHROMOSOMES, compareTrip);
	}
	\end{lstlisting}
	\pagebreak
	\subsection{Tsp.cpp}
	\begin{lstlisting}
	#include <iostream>  // cout
	#include <fstream>   // ifstream
	#include <string.h>  // strncpy
	#include <stdlib.h>  // rand
	#include <math.h>    // sqrt, pow
	#include <omp.h>     // OpenMP
	#include "Timer.h"
	#include "Trip.h"
	
	using namespace std;
	
	void initialize(Trip trip[CHROMOSOMES], int coordinates[CITIES][2]);
	
	void select(Trip trip[CHROMOSOMES], Trip parents[TOP_X]);
	
	void populate(Trip trip[CHROMOSOMES], Trip offsprings[TOP_X]);
	
	void formDisMax(int coordinates[CITIES][2], float disMax[CITIES][CITIES + 1]);
	
	extern void evaluate(Trip trip[CHROMOSOMES], int coordinates[CITIES][2]);
	
	extern void evaluateB(Trip trip[CHROMOSOMES], float disMax[CITIES][CITIES + 1], bool first);
	
	extern void crossover(Trip parents[TOP_X], Trip offsprings[TOP_X], int coordinates[CITIES][2]);
	
	extern void crossoverB(Trip parents[TOP_X], Trip offsprings[TOP_X], float disMax[CITIES][CITIES +
	1]);
	extern void mutate(Trip offsprings[TOP_X]);
	
	extern void mutateB(Trip offsprings[TOP_X], float disMax[CITIES][CITIES + 1]);
	
	/*
	* MAIN: usage: Tsp #threads
	*/
	int main(int argc, char *argv[]) {
	//Start random seed based on current time value
	srand(time(NULL));
	
	//static is required on windows machine
	static Trip trip[CHROMOSOMES];       // all 50000 different trips (or chromosomes)
	Trip shortest;                // the shortest path so far
	int coordinates[CITIES][2];   // (x, y) coordinates of all 36 cities:
	int nThreads = 1;
	float disMax[CITIES][CITIES + 1];
	
	// verify the arguments
	if (argc == 2)
	nThreads = atoi(argv[1]);
	else {
	cout << "usage: Tsp #threads" << endl;
	if (argc != 1)
	return -1; // wrong arguments
	}
	cout << "# threads = " << nThreads << endl;
	
	// shortest path not yet initialized
	shortest.itinerary[CITIES] = 0;  // null path
	shortest.fitness = -1.0;         // invalid distance
	
	// initialize 5000 trips and 36 cities' coordinates
	initialize(trip, coordinates);
	
	//from a distance matrix to improve efficiency
	formDisMax(coordinates, disMax);
	
	// start a timer
	Timer timer;
	timer.start();
	
	// change # of threads
	omp_set_num_threads(nThreads);
	
	// find the shortest path in each generation
	for (int generation = 0; generation < MAX_GENERATION; generation++) {
	// evaluate the distance of all 50000 trips
	evaluate(trip, coordinates, generation == 0);
	
	// just print out the progress
	if (generation % 20 == 0)
	cout << "generation: " << generation << endl;
	
	// whenever a shorter path was found, update the shortest path
	if (shortest.fitness < 0 || shortest.fitness > trip[0].fitness) {
	strncpy(shortest.itinerary, trip[0].itinerary, CITIES);
	shortest.fitness = trip[0].fitness;
	
	cout << "generation: " << generation
	<< " shortest distance = " << shortest.fitness
	<< "\t itinerary = " << shortest.itinerary << endl;
	}
	
	// define TOP_X parents and offsprings.
	// static is required on windows machine
	static Trip parents[TOP_X], offsprings[TOP_X];
	
	// choose TOP_X parents from trip
	select(trip, parents);
	
	// generates TOP_X offsprings from TOP_X parenets
	crossover(parents, offsprings, coordinates);
	
	// mutate offsprings
	mutateB(offsprings, disMax);
	// populate the next generation.
	populate(trip, offsprings);
	}
	
	// stop a timer
	cout << "elapsed time = " << timer.lap() << endl;
	return 0;
	}
	/**
	* Initializes trip[CHROMOSOMES] with chromosome.txt and coordiantes[CITIES][2] with cities.txt
	*
	* @param trip[CHROMOSOMES]:      50000 different trips
	* @param coordinates[CITIES][2]: (x, y) coordinates of 36 different cities: ABCDEFGHIJKLMNOPQRSTUVWXYZ
	*/
	void initialize(Trip trip[CHROMOSOMES], int coordinates[CITIES][2]) {
	// open two files to read chromosomes (i.e., trips)  and cities
	ifstream chromosome_file("chromosome.txt");
	ifstream cities_file("cities.txt");
	
	for (int i = 0; i < CHROMOSOMES; i++) {
	chromosome_file >> trip[i].itinerary;
	trip[i].fitness = 0.0;
	}
	
	for (int i = 0; i < CITIES; i++) {
	char city;
	cities_file >> city;
	int index = (city >= 'A') ? city - 'A' : city - '0' + 26;
	cities_file >> coordinates[index][0] >> coordinates[index][1];
	}
	
	chromosome_file.close();
	cities_file.close();
	
	if (DEBUG) {
	for (int i = 0; i < CHROMOSOMES; i++)
	cout << trip[i].itinerary << endl;
	for (int i = 0; i < CITIES; i++)
	cout << coordinates[i][0] << "\t" << coordinates[i][1] << endl;
	}
	}
	
	/**
	* Select the first TOP_X parents from trip[CHROMOSOMES]
	*
	* @param trip[CHROMOSOMES]: all trips
	* @param parents[TOP_X]:    the firt TOP_X parents
	*/
	void select(Trip trip[CHROMOSOMES], Trip parents[TOP_X]) {
	// just copy TOP_X trips to parents
	for (int i = 0; i < TOP_X; i++)
	strncpy(parents[i].itinerary, trip[i].itinerary, CITIES + 1);
	}
	
	/**
	* Replace the bottom TOP_X trips with the TOP_X offsprings
	*/
	void populate(Trip trip[CHROMOSOMES], Trip offsprings[TOP_X]) {
	// just copy TOP_X offsprings to the bottom TOP_X trips.
	for (int i = 0; i < TOP_X; i++) {
	strncpy(trip[CHROMOSOMES - TOP_X + i].itinerary, offsprings[i].itinerary, CITIES);
	}
	// for debugging
	if (DEBUG) {
	for (int chrom = 0; chrom < CHROMOSOMES; chrom++)
	cout << "chrom[" << chrom << "] = " << trip[chrom].itinerary
	<< ", trip distance = " << trip[chrom].fitness << endl;
	}
	}
	
	/**
	* Generate CITIES distances for each city, total CITIES * CITIES distances
	* @param coordinates (x, y) coordinates of CITIES different cities
	* @param disMax Distance matrix
	*/
	void formDisMax(int coordinates[CITIES][2], float disMax[CITIES][CITIES + 1]) {
	for (int i = 0; i < CITIES; ++i) {
	for (int j = 0; j < CITIES; ++j) {
	disMax[i][j] = hypot((coordinates[i][0] - coordinates[j][0]),
	(coordinates[i][1] - coordinates[j][1]));
	}
	disMax[i][CITIES] = hypot(coordinates[i][0] , coordinates[i][1]);
	}
	}
	\end{lstlisting}
	
	
	
	\section {Execution output}
	\subsection{Output analysis}
	\begin{enumerate} 
		\item The shortest trip in my program is equal to 447.638
		\item The performance improvement with four threads in my program is equal to 50548672 / 23005048 = 2.2 times
	\end{enumerate}
	
	\subsection{Output}
	\begin{small}
		\noindent \# threads = 1 \newline
		generation: 0 \newline
		generation: 0 shortest distance = 1265.72	 itinerary = V1SPMBQAN26G4J37DX8OTF95ZUH0EYRLCWKI
		generation: 1 shortest distance = 1031.81	 itinerary = I61YHO9F48KGATL7UJR3BQ20ENXVSCZWP5MD
		generation: 2 shortest distance = 979.47	 itinerary = KFL94NT86OJAGX7RD3IU2W5PBS10EZHYVCMQ
		generation: 3 shortest distance = 882.109	 itinerary = V1O6XG84K9AFLPDBM3Q7NTIUZJE02HYCSW5R
		generation: 4 shortest distance = 724.804	 itinerary = V1YHEJUZ02CSW5MBQDR794X6IONTK8FALG3P
		generation: 5 shortest distance = 627.852	 itinerary = V1YHEZ02CSW5MPBQDR3JU7ALGF4NT89KX6IO
		generation: 7 shortest distance = 626.585	 itinerary = V1IO6JE02WCS5HZYFAL7R3DBQMPUKX4NT8G9
		generation: 8 shortest distance = 603.274	 itinerary = V1YHEUZJWSC205MPQ3RDB7LAF9KGX48TN6OI
		generation: 10 shortest distance = 556.14	 itinerary = V1YZHUE0MSWC5DBQR37LA9KFGXNT48IO6J2P
		generation: 11 shortest distance = 542.062	 itinerary = V1YZHUE0MPBQR37LA9KFGXNT48IO6J2SWC5D
		generation: 13 shortest distance = 477.948	 itinerary = V1YZHUE02WSC5MPBQDR37LA9KFGXNT48IO6J
		generation: 16 shortest distance = 474.401	 itinerary = V1YZHUE02WSC5MPBQDR37LAFK9GXNT48IO6J
		generation: 18 shortest distance = 467.935	 itinerary = V1YZHUE0J6OI84NTGXKF9AL7R3DBQPMSWC52
		generation: 19 shortest distance = 467.25	 itinerary = V1YZH5CWSMPQBDR37LAF9KGXNT48IO6J0E2U
		generation: 20\newline
		generation: 20 shortest distance = 464.471	 itinerary = V1YZHUE20J6OI84NTGX9FKAL7R3DBQPMWSC5
		generation: 24 shortest distance = 461.723	 itinerary = V1YZHUE025CWSMPQBDR37LA9KFGXNT48IO6J
		generation: 30 shortest distance = 459.941	 itinerary = V1YZHUE20J6OI84NTGX9FKAL7R3DBQPMSWC5
		generation: 33 shortest distance = 459.436	 itinerary = V1YZH5CWSMPQBDR37LAF9KGXNT48IO6JUE20
		generation: 35 shortest distance = 458.176	 itinerary = V1YZHUE025CWSMPQBDR37LAFK9GXNT48IO6J
		generation: 37 shortest distance = 457.536	 itinerary = V1YZH5CWSMPQBDR37LAF9KGXNT48IO6J02EU
		generation: 40\newline
		generation: 43 shortest distance = 456.303	 itinerary = V1YZHUE20J6OI84NTGX9KFAL7R3DBQPMSWC5
		generation: 44 shortest distance = 452.975	 itinerary = V1YZHUE025CWSMPQBDR37LAF9KGXNT48IO6J
		generation: 54 shortest distance = 449.658	 itinerary = V1YZHUE025CWSMPQBD3R7LAF9KGXNT48IO6J
		generation: 60\newline
		generation: 61 shortest distance = 449.552	 itinerary = V1YZHUE20J6OI84NTXGK9FAL7R3DBQPMSWC5
		generation: 71 shortest distance = 447.638	 itinerary = V1YZHUE20J6OI84TNXGK9FAL7R3DBQPMSWC5
		generation: 80\newline
		generation: 100\newline
		generation: 120\newline
		generation: 140\newline
		elapsed time = 50548672
		
		
		\noindent \# threads = 4 \newline
		generation: 0 \newline
		generation: 0 shortest distance = 1265.72	 itinerary = V1SPMBQAN26G4J37DX8OTF95ZUH0EYRLCWKI
		generation: 1 shortest distance = 1043.94	 itinerary = I61YHO9F48KGATL7UJR3BQ2P5SCZW0ENXVMD
		generation: 2 shortest distance = 871.976	 itinerary = VU25BSCWMPQDR37JZHT8IONX9AF46K0EY1GL
		generation: 3 shortest distance = 825.757	 itinerary = V8TNKA9FXGIO1HZYUE5C6RL73DMBQPSW2J04
		generation: 4 shortest distance = 805.385	 itinerary = 1VKF9X7AGTJ6OI4N83RL0SWC25BPMQDEUHYZ
		generation: 5 shortest distance = 716.93	 itinerary = 8TN4X9KGAF6OIJ02EHYZU1VSCWM5BDL7R3QP
		generation: 7 shortest distance = 652.966	 itinerary = 1V6OI9KFAL7J0WC5SMPBQ3RDHZYUE284GXNT
		generation: 8 shortest distance = 635.406	 itinerary = V1YIO6NTXK948GR3DBPMS5CW02ZHUEJ7LFAQ
		generation: 9 shortest distance = 583.042	 itinerary = V1YZUH5CSWMPBQD3R7AF9KTNX4GOI68L2E0J
		generation: 10 shortest distance = 564.252	 itinerary = V1YHUZ5CWSMPQBDR37LJOI4NT89FKXGA6E02
		generation: 11 shortest distance = 558.654	 itinerary = UHZY1VO6I8KGXNT49FAL7R3QBPMSWC520EJD
		generation: 12 shortest distance = 545.421	 itinerary = V1YZHUE20J6OI9FK84NTGXAL73RDBQSWCMP5
		generation: 13 shortest distance = 536.668	 itinerary = V1YZHUE20J6OI4N8T9XGKFAL73R5CWSMPQBD
		generation: 14 shortest distance = 520.507	 itinerary = V1YZHU5CWSE0J6OI84NTXG9KFAL7R3DBPM2Q
		generation: 15 shortest distance = 496.876	 itinerary = V1YZHUE0J6OI84NTXGK9FLA7R325CWSMPBDQ
		generation: 16 shortest distance = 496.376	 itinerary = V1YZHUE02WSC5MPQBD3R7LAGXKF984NTJ6OI
		generation: 17 shortest distance = 484.545	 itinerary = V1YZHUE02J6OI84NT9FKXGAL7R3QBDPMSWC5
		generation: 18 shortest distance = 469.679	 itinerary = V1YZHUE025CWSMPQBDR37LAFKGXNT849IO6J
		generation: 20\newline
		generation: 22 shortest distance = 467.855	 itinerary = V1YZHUE02J6OI84NT9FKXGAL7R3DBQPMSWC5
		generation: 24 shortest distance = 460.906	 itinerary = V1YZHUE20J6OI84NT9FKXGAL7R3DBQPMSWC5
		generation: 29 shortest distance = 460.657	 itinerary = V1YZHUE02J6OI84NTXGKF9AL7R3DBQPMSWC5
		generation: 36 shortest distance = 453.709	 itinerary = V1YZHUE20J6OI84NTXGKF9AL7R3DBQPMSWC5
		generation: 40\newline
		generation: 43 shortest distance = 449.552	 itinerary = V1YZHUE20J6OI84NTXGK9FAL7R3DBQPMSWC5
		generation: 60\newline
		generation: 60 shortest distance = 447.638	 itinerary = V1YZHUE20J6OI84TNXGK9FAL7R3DBQPMSWC5
		generation: 80\newline
		generation: 100\newline
		generation: 120\newline
		generation: 140\newline
		elapsed time = 23005048
	\end{small}
	
	
	\section {Discussions}
	In addition, I tried to improve the whole efficiency by replacing calculating distance with a cached matrix that contains all the 36 * 36 distance. This implementation decrease significant amount of time spent, although this made the program program with single thread runs faster than that with multi-thread. This also explained why select() with multi-thread could be slower than that with single thread. Starting a multi-thread could require some time to analysis the for-loop and distribute the tasks. Therefore, it is best to use multi-thread when the loop is large and require large computational power. \par To improve the performance of current program, just use the implementation I mentioned above to shorten the time required to calculate the distance. However, this only works with the current data set, it could be less efficient with larger data sets.
	
	\section {Lab Sessions 1}
	Lab 1 we parallelize two programs that compute Pi using Monte Carlo methods and Integration. As the result shown, multi-thread decrease significant amount of time that required to finish the program. However, if the number of iterations is too small, the performance would decrease.
	
	\subsection{Execution output }
	\noindent \large pi\_integral\_omp
	
	
	\begin{small}	
		
		\noindent Enter the number of iterations used to estimate pi: 1000000000\newline
		Enter the number of threads: 1\newline
		elapsed time for pi = 10850791\newline
		\# of trials = 1000000000, estimate of pi is  3.1415926535899708, Error is 0.0000000000001776\newline
		
		\noindent Enter the number of iterations used to estimate pi: 1000000000 \newline
		Enter the number of threads: 4 \newline
		elapsed time for pi = 2831500 \newline
		\# of trials = 1000000000, estimate of pi is  3.1415926535898211, Error is 0.0000000000000280 \newline
		
	\end{small}
	
	\noindent \large pi\_monte\_omp
	
	\begin{small}	
		
		\noindent Enter the number of iterations used to estimate pi: 1000000000\newline
		Enter the number of threads: 1\newline
		elapsed time for pi = 44118944\newline
		\# of trials = 1000000000, estimate of pi is  3.1416131730000001, Error is 0.0000205194102070\newline
		
		\noindent Enter the number of iterations used to estimate pi: 1000000000\newline
		Enter the number of threads: 4\newline
		elapsed time for pi = 14253004\newline
		\# of trials = 1000000000, estimate of pi is  3.1415967750000000, Error is 0.0000041214102069\newline
		
	\end{small}
	
\end{document}


















