\documentclass[11pt, letterpaper]{article}
\usepackage{color}
\usepackage[linktoc=all]{hyperref}

\usepackage[bindingoffset=0.2in,left=1in,right=1in,top=0.5in,bottom=1in,footskip=.25in]{geometry}
\usepackage{enumitem}
\usepackage{graphicx}
\usepackage{subcaption}
\usepackage{mwe}
\usepackage{parskip}
\usepackage{indentfirst}
\usepackage{textcomp}
\usepackage[formats]{listings}
\usepackage{xcolor}
\definecolor{listinggray}{gray}{0.9}
\definecolor{lbcolor}{rgb}{0.9,0.9,0.9}
\definecolor{Darkgreen}{RGB}{11,100,35}
\usepackage{geometry}

\geometry{
	body={7.0in, 9.8in},
	left=0.75in,
	top=0.6in
}

\hypersetup{
	colorlinks,
	citecolor=black,
	filecolor=black,
	linkcolor=black,
	urlcolor=black
}

\lstset{
	backgroundcolor=\color{lbcolor},
	tabsize=4,    
	%   rulecolor=,
	language=[GNU]C++,
	basicstyle=\scriptsize,
	upquote=true,
	aboveskip={1.5\baselineskip},
	columns=fixed,
	showstringspaces=false,
	extendedchars=false,
	breaklines=true,
	prebreak = \raisebox{0ex}[0ex][0ex]{\ensuremath{\hookleftarrow}},
	frame=single,
	numbers=left,
	showtabs=false,
	showspaces=false,
	showstringspaces=false,
	identifierstyle=\ttfamily,
	keywordstyle=\color[rgb]{0,0,1},
	commentstyle=\color[rgb]{0.026,0.112,0.095},
	stringstyle=\color[rgb]{0.627,0.126,0.941},
	numberstyle=\color[rgb]{0.205, 0.142, 0.73},
	%        \lstdefinestyle{C++}{language=C++,style=numbers}’.
}



\lstset{
	backgroundcolor=\color{lbcolor},
	tabsize=4,
	language=C++,
	captionpos=b,
	tabsize=3,
	frame=lines,
	numbers=left,
	numberstyle=\tiny,
	numbersep=5pt,
	breaklines=true,
	showstringspaces=false,
	basicstyle=\footnotesize,
	%  identifierstyle=\color{magenta},
	keywordstyle=\color[rgb]{0,0,1},
	commentstyle=\color{Darkgreen},
	stringstyle=\color{red}
}



\setlength{\parindent}{15pt}
%opening
\title{\textbf{Program 2 Report}}
\author{Yangxiao Wang}
\date{ }



\begin{document}
	
	\maketitle
	
	\tableofcontents
	\pagebreak
	
	\section{Documentation}
	In this project, we tried to build a two-dimensional wave diffusion program by using Schroedinger's Wave Dissemination formula. Since the formula requires heavy computational power, we proposed to use parallelization techniques to implement it. And in this report, I would demonstrate the code and time required for each of the following implementations: sequential, MPI, and hybrid form of MPI and OpenMPI.	\par
	The MPI methods I used is the two blocking communications \textit{MPI\_Send()} and \textit{MPI\_Recv()}. The advantage of using blocking communication is I do not need to handle the synchronization problem. The program can only continue to work until all communication is done (all data is sent and received). In addition, to exchange the boundary data, my approach was sending the boundary stripe at beginning of every loop of $t$ (time). And for implementation of OpenMPI, I simply parallelized the Schroedinger's formula part. For each ranks, it has multiple threads to compute the formula.
	
	
	\section {Source code}
	\subsection{Wave2D.cpp}
	\vspace{-0.2in}
	\begin{lstlisting}
	#include <iostream>
	#include "Timer.h"
	#include <stdlib.h>   // atoi
	#include <math.h>
	#include <stdio.h>
	
	int default_size = 100;  // the default system size
	int defaultCellWidth = 8;
	double c = 1.0;      // wave speed
	double dt = 0.1;     // time quantum
	double dd = 2.0;     // change in system
	
	using namespace std;
	
	int main(int argc, char *argv[]) {
		// verify arguments
		if (argc != 4) {
			cerr << "usage: Wave2D size max_time interval" << endl;
			return -1;
		}
		int size = atoi(argv[1]);
		int max_time = atoi(argv[2]);
		int interval = atoi(argv[3]);
		
		if (size < 100 || max_time < 3 || interval < 0) {
			cerr << "usage: Wave2D size max_time interval" << endl;
			cerr << "       where size >= 100 && time >= 3 && interval >= 0" << endl;
			return -1;
		}
		
		// create a simulation space
		double z[3][size][size];
		for (int p = 0; p < 3; p++)
			for (int i = 0; i < size; i++)
				for (int j = 0; j < size; j++)
					z[p][i][j] = 0.0; // no wave
		
		// start a timer
		Timer time;
		time.start();
		
		// time = 0;
		// initialize the simulation space: calculate z[0][][]
		int weight = size / default_size;
		for (int i = 0; i < size; i++) {
			for (int j = 0; j < size; j++) {
				if (i > 40 * weight && i < 60 * weight &&
					j > 40 * weight && j < 60 * weight) {
					z[0][i][j] = 20.0;
				} else {
					z[0][i][j] = 0.0;
				}
			}
		}
		
		// time = 1
		for (int i = 1; i < size - 1; i++) {
			for (int j = 1; j < size - 1; j++) {
				z[1][i][j] = z[0][i][j] + (pow(c, 2) / 2) * pow(dt / dd, 2) * (z[0][i + 1][j] + z[0][i - 1][j] + z[0][i][j + 1] + z[0][i][j - 1] - (4.0 * z[0][i][j]));
			}
		}
		
		// simulate wave diffusion from time = 2
		for (int t = 2; t < max_time; t++) {
			int time = t % 3;
			for (int i = 1; i < size - 1; i++) {
				for (int j = 1; j < size - 1; j++) {
					int time_1, time_2;
					//rotate z
					if (time == 0) {
						time_1 = 2;
						time_2 = 1;
					} else if (time == 1) {
						time_1 = 0;
						time_2 = 2;
					} else {
						time_1 = 1;
						time_2 = 0;
					}
					//calculation
					z[time][i][j] =
					2.0 * z[time_1][i][j] - z[time_2][i][j] + (pow(c, 2) * pow(dt / dd, 2)	* (z[time_1][i + 1][j] + z[time_1][i - 1][j] + z[time_1][i][j + 1] +	z[time_1][i][j - 1] - (4.0 * z[time_1][i][j])));
				}
			}
			//print out
			if (interval != 0 && t % interval == 0) {
				cout << t << endl;
				for (int j = 0; j < size; j++) {
					for (int i = 0; i < size; i++) {
						cout << z[time][i][j] << " ";
					}
					cout << endl;
				}
				cout << endl;
			}
		} // end of simulation
		
		// finish the timer
		cerr << "Elapsed time = " << time.lap() << endl;
		return 0;
	}
	\end{lstlisting}
	\pagebreak
	
	\subsection{Wave2D mpi.cpp}
	\vspace{-0.2in}
	\begin{lstlisting}
	#include <iostream>
	#include "Timer.h"
	#include <stdlib.h>   // atoi
	#include <math.h>
	#include <mpi.h>
	#include <stdio.h>
	#include <omp.h>
	
	int default_size = 100;  // the default system size
	int defaultCellWidth = 8;
	double c = 1.0;      // wave speed
	double dt = 0.1;     // time quantum
	double dd = 2.0;     // change in system
	
	using namespace std;
	
	int main(int argc, char *argv[]) {
		int my_rank = 0;            // used by MPI
		// used by MPI
		
		// verify arguments
		if (argc != 5) {
			cerr << "usage: Wave2D size max_time interval n_thread" << endl;
			return -1;
		}
		int size = atoi(argv[1]);
		int max_time = atoi(argv[2]);
		int interval = atoi(argv[3]);
		int nThreads = atoi(argv[4]);
		int mpi_size;
		
		if (size < 100 || max_time < 3 || interval < 0 || nThreads <= 0) {
			cerr << "usage: Wave2D size max_time interval" << endl;
			cerr << "where size >= 100 && time >= 3 && interval >= 0 && mpi_size > 0" << endl;
			return -1;
		}
		
		// start MPI
		MPI_Init(&argc, &argv);
		MPI_Comm_rank(MPI_COMM_WORLD, &my_rank);
		MPI_Comm_size(MPI_COMM_WORLD, &mpi_size);
		// change # of threads for openMP
		omp_set_num_threads(nThreads);
		
		// create a simulation space
		double z[3][size][size];
		for (int p = 0; p < 3; p++)
			for (int i = 0; i < size; i++)
				for (int j = 0; j < size; j++)
					z[p][i][j] = 0.0; // no wave
		
		// start a timer
		Timer time;
		time.start();
		
		// time = 0;
		// initialize the simulation space: calculate z[0][][]
		int weight = size / default_size;
		for (int i = 0; i < size; i++) {
			for (int j = 0; j < size; j++) {
				if (i > 40 * weight && i < 60 * weight && j > 40 * weight && j < 60 * weight) {
					z[0][i][j] = 20.0;
				} else {
					z[0][i][j] = 0.0;
				}
			}
		}
		
		// time = 1, and parallelization
		#pragma omp parallel for
		for (int i = 1; i < size - 1; i++) {
			for (int j = 1; j < size - 1; j++) {
				z[1][i][j] = z[0][i][j] + (pow(c, 2) / 2) * pow(dt / dd, 2) * (z[0][i + 1][j] + z[0][i - 1][j] + z[0][i][j + 1] + z[0][i][j - 1] - (4.0 * z[0][i][j]));
			}
		}
		
		
		int stripe = size / mpi_size;     // partitioned stripe
		
		// simulate wave diffusion from time = 2
		for (int t = 2; t < max_time; t++) {
			//rotate z
			int time = t % 3;
			int time_1, time_2;
			if (time == 0) {
				time_1 = 2;
				time_2 = 1;
			} else if (time == 1) {
				time_1 = 0;
				time_2 = 2;
			} else {
				time_1 = 1;
				time_2 = 0;
			}
		
			//exchange boundary data
			if (my_rank == 0) {
				MPI_Send(*(*(z + time_1) + stripe * (my_rank + 1) - 1), size, MPI_DOUBLE, my_rank + 1, 0, MPI_COMM_WORLD);
				
				MPI_Status status;
				MPI_Recv(*(*(z + time_1) + stripe), size, MPI_DOUBLE, my_rank + 1, 0, MPI_COMM_WORLD, &status);
			} else if (my_rank == mpi_size - 1) {
				MPI_Send(*(*(z + time_1) + stripe * my_rank), size, MPI_DOUBLE, my_rank -		1, 0, MPI_COMM_WORLD);
				
				MPI_Status status;
				MPI_Recv(*(*(z + time_1) + stripe * my_rank - 1), size, MPI_DOUBLE, my_rank - 1, 0, MPI_COMM_WORLD, &status);
			} else {
				MPI_Send(*(*(z + time_1) + stripe * my_rank), size, MPI_DOUBLE, my_rank - 1, 0,	MPI_COMM_WORLD);
				MPI_Send(*(*(z + time_1) + stripe * (my_rank + 1) - 1), size, MPI_DOUBLE, my_rank + 1, 0, MPI_COMM_WORLD);
				
				MPI_Status status;
				MPI_Recv(*(*(z + time_1) + stripe * my_rank - 1), size, MPI_DOUBLE, my_rank - 1, 0, MPI_COMM_WORLD, &status);
				MPI_Recv(*(*(z + time_1) + stripe * (my_rank + 1)), size, MPI_DOUBLE, my_rank + 1, 0, MPI_COMM_WORLD, &status);
			}
			
			//Parallelization for the Schroedinger's formula
			#pragma omp parallel for
			for (int i = my_rank * stripe; i < (my_rank + 1) * stripe; i++) {
				if (i == 0 || i == size - 1) {
					continue;
				}
				for (int j = 1; j < size - 1; j++) {
				
					z[time][i][j] =
					2.0 * z[time_1][i][j] - z[time_2][i][j] + (pow(c, 2) * pow(dt / dd, 2)
					* (z[time_1][i + 1][j] + z[time_1][i - 1][j] + z[time_1][i][j + 1] +
					z[time_1][i][j - 1] - (4.0 * z[time_1][i][j])));
				}
			}
				
			//output if it's interval
			if (interval != 0 && t % interval == 0) {
				//Aggregate all results from all ranks
				if (my_rank == 0) {
					for (int rank = 1; rank < mpi_size; ++rank) {
						MPI_Status status;
						MPI_Recv(*(*(z + time) + rank * stripe), stripe * size, MPI_DOUBLE, rank, 0, MPI_COMM_WORLD, &status);
					}
				
					cout << t << endl;
					for (int j = 0; j < size; j++) {
						for (int i = 0; i < size; i++) {
							cout << z[time][i][j] << " ";
						}
						cout << endl;
					}
					cout << endl;
				
				} else {
					MPI_Send(*(*(z + time) + my_rank * stripe), stripe * size, MPI_DOUBLE, 0, 0,
					MPI_COMM_WORLD);
				}
			}
		} // end of simulation
		
		MPI_Finalize(); // shut down MPI
		
		// finish the timer
		if(my_rank == 0) {
			cerr << "Elapsed time = " << time.lap() << endl;
		}
		
		return 0;
	}
	\end{lstlisting}
	
	
	
	\section {Execution output}
	\subsection{Output analysis}
	\begin{enumerate} 
		\item The performance improvement with four machines: 5721050 / 1575955 = 3.63 times
		\item The performance improvement with four machines with multithreading: 5721050 / 874590 = 6.54 times
	\end{enumerate}
	
	\subsection{Execution output}
	\noindent \large Check if output is correct
	\vspace{-0.2in}
	\begin{lstlisting}
	[wyxiao_css534@cssmpi1 prog2]$ ./Wave2D 576 500 50 > reS.txt
	Elapsed time = 7313114
	[wyxiao_css534@cssmpi1 prog2]$ mpirun -n 4 ./Wave2D_mpi 576 500 50 4 > reF.txt
	Elapsed time = 6991195
	[wyxiao_css534@cssmpi1 prog2]$ diff reF.txt reS.txt
	[wyxiao_css534@cssmpi1 prog2]$
	\end{lstlisting}
	
	\noindent \large Check output the performance improvement with four machines: 5721050 / 1575955 = 3.63 times
	\vspace{-0.2in}
	\begin{lstlisting}
	[wyxiao_css534@cssmpi1 prog2]$ ./Wave2D 576 500 0
	Elapsed time = 5721050
	[wyxiao_css534@cssmpi1 prog2]$ mpirun -n 4 ./Wave2D_mpi 576 500 0 1
	Elapsed time = 1575955
	\end{lstlisting}
	
	\noindent \large Check output the performance improvement with four machines with multithreading: 5721050 / 874590 = 6.54 times
	\vspace{-0.2in}
	\begin{lstlisting}
	[wyxiao_css534@cssmpi1 prog2]$ mpirun -n 4 ./Wave2D_mpi 576 500 0 4
	Elapsed time = 874590
	\end{lstlisting}
	
	\section {Discussions}
	I noticed the cost of communication is not ideal, especially the aggregation of all ranks' data. Therefore I should keep the numbers of aggregation as low as possible, the aggregation only happens when the output is required.\par
	To improve the current implementation, the best way I could think of is using unblocking communication (\textit{MPI\_Isend} and \textit{MPI\_Irecv}) when aggregation happens. Then the program would not stop when perform aggregation. And add a \textit{MPI\_Wait} statement before $t = t_{current}+3$ to make sure previous communication is finished when changing the values. Since the aggregation is the most time consuming part.
	
	\section {Lab Sessions 2}
	Lab 2 we parallelize a programs that compute the result of multiplication of matrix using sequential and MPI. As the result shown, four MPI ranks decrease significant amount of time that required to finish the program. However, if the size of matrix is too small, the performance would decrease as shown in the first execution output.
	
	\subsection{Source Code}
	\vspace{-0.2in}
	\begin{lstlisting}
	#include "mpi.h"
	#include <stdlib.h> // atoi
	#include <iostream> // cerr
	#include "Timer.h"
	
	using namespace std;
	
	void init(double *matrix, int size, char op) {
		for (int i = 0; i < size; i++)
			for (int j = 0; j < size; j++)
				matrix[i * size + j] = (op == '+') ? i + j : ((op == '-') ? i - j : 0);
	}
	
	void print(double *matrix, int size, char id) {
		for (int i = 0; i < size; i++)
			for (int j = 0; j < size; j++)
				cout << id << "[" << i << "][" << j << "] = " << matrix[i * size + j] << endl;
	}
	
	void multiplication(double *a, double *b, double *c, int stripe, int size) {
		for (int k = 0; k < size; k++)
			for (int i = 0; i < stripe; i++)
				for (int j = 0; j < size; j++)
					// c[i][k] += a[i][j] * b[j][k];
					c[i * size + k] += a[i * size + j] * b[j * size + k];
	}
	
	int main(int argc, char *argv[]) {
		int my_rank = 0;            // used by MPI
		int mpi_size = 1;           // used by MPI
		int size = 400;             // array size
		bool print_option = false;  // print out c[] if it is true
		Timer timer;
		
		// variables verification
		if (argc == 3) {
		if (argv[2][0] == 'y')
			print_option = true;
		}
		
		if (argc == 2 || argc == 3) {
			size = atoi(argv[1]);
		} else {
			cerr << "usage:   matrix size [y|n]" << endl;
			cerr << "example: matrix 400   y" << endl;
			return -1;
		}
		
		MPI_Init(&argc, &argv); // start MPI
		MPI_Comm_rank(MPI_COMM_WORLD, &my_rank);
		MPI_Comm_size(MPI_COMM_WORLD, &mpi_size);
		
		// matrix initialization
		double *a = new double[size * size];
		double *b = new double[size * size];
		double *c = new double[size * size];
		
		if (my_rank == 0) { // master initializes all matrices
			init(a, size, '+');
			init(b, size, '-');
			init(c, size, '0');
			
			// print initial values
			if (false) {
				print(a, size, 'a');
				print(b, size, 'b');
			}
		
			// start a timer
			timer.start();
		} else {                // slavs zero-initializes all matrices
			init(a, size, '0');
			init(b, size, '0');
			init(c, size, '0');
		}
		
		// broadcast the matrix size to all.
		MPI_Bcast(&size, 1, MPI_INT, 0, MPI_COMM_WORLD);
		
		int stripe = size / mpi_size;     // partitioned stripe
		
		// master sends each partition of a[] to a different slave
		// master also sends b[] to all slaves
		if (my_rank == 0) {
			for (int rank = 1; rank < mpi_size; ++rank) {
				MPI_Send(a + rank * stripe * size, size * stripe, MPI_DOUBLE, rank, 0, MPI_COMM_WORLD);
				MPI_Send(b, size * size, MPI_DOUBLE, rank, 0, MPI_COMM_WORLD);
			}
		} else {
			MPI_Status status;
			MPI_Recv(a, size * stripe, MPI_DOUBLE, 0, 0, MPI_COMM_WORLD, &status);
			MPI_Recv(b, size * size, MPI_DOUBLE, 0, 0, MPI_COMM_WORLD, &status);
		}
		
		multiplication(a, b, c, stripe, size); // all ranks should compute multiplication
		
		// master receives each partition of c[] from a different slave
		if (my_rank == 0) {
			for (int rank = 1; rank < mpi_size; ++rank) {
				MPI_Status status;
				MPI_Recv(c + rank * stripe * size, size * stripe, MPI_DOUBLE, rank, 0, MPI_COMM_WORLD,
				&status);
			}
		} else {
			MPI_Send(c, stripe * size, MPI_DOUBLE, 0, 0, MPI_COMM_WORLD);
		}
		
		if (my_rank == 0)
			// stop the timer
			cout << "elapsed time = " << timer.lap() << endl;
		
		// results
		if (print_option && my_rank == 0)
			print(c, size, 'c');
		
		MPI_Finalize(); // shut down MPI
	}
	\end{lstlisting}
	
	\subsection{Execution output}
	
	\noindent \large Check output is correct
	\vspace{-0.2in}
	\begin{lstlisting}
	[wyxiao_css534@cssmpi1 lab2]$ ./matrix 100 y > maxS.txt
	[wyxiao_css534@cssmpi1 lab2]$ mpirun -n 4 ./matrix_mpi 100 y > maxM.txt
	[wyxiao_css534@cssmpi1 lab2]$ diff maxS.txt maxM.txt
	1c1
	< elapsed time = 5185
	---
	> elapsed time = 7133
	[wyxiao_css534@cssmpi1 lab2]$
	\end{lstlisting}
	
	\noindent \large Check output the performance improvement: 774693 / 258639 = 2.9952 times
	\vspace{-0.2in}
	\begin{lstlisting}
	[wyxiao_css534@cssmpi1 lab2]$ ./matrix 500
	elapsed time = 774693
	[wyxiao_css534@cssmpi1 lab2]$ mpirun -n 4 ./matrix_mpi 500
	elapsed time = 258639
	[wyxiao_css534@cssmpi1 lab2]$
	\end{lstlisting}
	
	
\end{document}


















